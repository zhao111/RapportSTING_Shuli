\section{Dégagement des contraintes de développement}

Selon des fonctions définies et des codes existantes, j'ai défini des contraintes et proposé des solutions :
\begin{enumerate}
    \item La gestion d'utilisateurs
    
    \textbf{Contraintes} 
    \begin{itemize}
        \item Manque d'un système d'accès connecté celui de l’écoles
        \item Sans reconnaissance d’identités
    \end{itemize}
    \textbf{Propositions} 
    \begin{itemize}
        \item Communiquer avec les responsables informatiques d'écoles pour intérroger des méthodes de connection
        \item Réaliser la reconnaissance selon les mails address \footnote{ Par exemple, @eleves.ec-nantes.fr indique un élève et @ec-nantes.fr indique un enseignant } 
        \item Avant l'intégration de comptes, un système d'autorisation provisoire ressemblant celui d'ezb sera réalisé
        \item Des messageries à ajouter supplémentairement à la fin
    \end{itemize}

    \item La gestion de livres

    \textbf{Contraintes} 
    \begin{itemize}
        \item Manque de possibilités d'utilisation de multi-format et téléchargement
    \end{itemize} 
    \textbf{Propositions} 
    \begin{itemize}
        \item Sauvegarder des fichiers en format XML, TEI précisement, pour sa utilisation générale dans le domaine de livre numérique
        \item Chercher des outils de transformations, sinon il faut fabriquer un 
        \item Permettre le créateur d'un livre de définir son status, publique ou privé, pour éviter des problèmes de copyright
    \end{itemize}

    \item La gestion de projets

    \textbf{Contraintes}
    \begin{itemize}
        \item Impossibilité de définir la finalité par utilisateurs et de les distribuer aux sous-groupes
        \item Difficulté d'organiser des citations, résumés, écritures créatives, etc dans un layer
        \item Manque des contrôles de layers et des possibilité d'inviter des autres éditeurs hors du groupe
    \end{itemize}
    \textbf{Propositions} 
    \begin{itemize}
        \item Introduire les conception \textbf{Marqueur}, qui indique une position dans le livre, et \textbf{Séquence}, qui permet de choisir un marqueur au début, un à la fin et des élèves d'édition
        \item Limiter le nombre de niveaux de layers, définir trois type de layers 
            \begin{itemize}
                \item \textbf{Layer Original } qui est une copie d'un livre et on peut définir des marqueurs et des séquences dessus
                \item \textbf{Layer Travail } qui permet d'ajouter soit des citations soit des éctiture \footnote{ traduction, résumés, écritures créatives, ect... } 
                \item \textbf{Layer Commentaire } qui correspond automatiquement aux travails dans le layer parent
            \end{itemize}
        \item Permettre de choisir la finalité comprenant celle définie par utilisateur, mais aussi d'entier livre ou de chapitres ou de paragraphes fournies automatiquement par la plate-forme, pour chaque layer travail pour réaliser différent type d'écriture \footnote{ Par exemple, utiliser la finalité de paragraphes pour la traduction, utiliser la finalité de chapitres pour les résumés de chapitres, etc...} 
        \item Possible de choisir Privé comme le status d'un layer qui sera invisible pour le publique ni copi-able
        \item Permettre d'inviter des autres personnes pour contribuer dans le projet ou commenter des travails faits par des élèves
    \end{itemize}

    \item La gestion de groupes

    \textbf{Contraintes}
    \begin{itemize}
        \item Trop de niveaux à choisir quand ajouter un membre qui comprend même le niveau Créateur
    \end{itemize}
    \textbf{Propositions} 
    \begin{itemize}
        \item Ajouter des membres sans définir leur rôle, ne choisir qu'un seul membre comme le manager du groupe 
    \end{itemize}   
\end{enumerate}