\chapter*{Introduction}
\addcontentsline{toc}{chapter}{Introduction}

Avec le développement de numérisation et demande de travail collaboration, les enseignements de lecture ont besoin des outils qui permettent de travailler sur un livre de la manière synchrone à plusieurs niveaux détaillés. 

Dans le cadre de mes études de l'informatique, j'ai souhaité réalisé mon stage sur un projet répondant à ces enjeux pédagogiques en matière de développement d'une plate-forme web. Les missions de l'évaluation de l'évolution existante et de la proposition des solutions pour des dégagement des contraintes de développement m'ont attirés particulièrement car je souhaitais savoir la démarche d'un projet et si ce type de métier pouvait m'intéresser alors que je m'oriente dans ma formation et professionnellement vers le management de projet. 

Aussi la \pe\ s'est fait connaitre avec succès pour ses lancements de différents projets avec le but de la contribution dans le domaine de l'enfance et de la jeunesse, j'ai voulu intégrer ses équipes pour pouvoir découvrir des chercheurs et leurs recherches. Particulièrement, j'avait travaillé sur le projet \ezb\ pendant le projet d'application qui me donnait déjà des connaissances de ses actions principes.

Pendant les 20 semaines de mon stage, j’ai traversé 6 périodes principales dans lesquelles j’ai rencontré différents types de problèmes qui me donnent une expérience très riche. 

Ce rapport est organisé selon les 6 périodes dont je vais décrier la difficulté, des problèmes à résoudre, la démarche, des résultat et la conclusion. J’espère qu’il puisse montrer mes expériences clairement. 