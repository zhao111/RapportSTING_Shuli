\section{Management de projet}

A part les technologies, il y avait aussi une grande partie de travail qui m'a donné les expériences du management de projet.

Au début, j'avais pensé que le codes ait été la partie plus importante dans un projet, mais après les discussions avec les enseignant-chercheurs pour définir la conception, les fonctions et les interfaces de la plateforme, j'ai trouvé que une conception bien définie est la fondation du développement. 

Il faut prendre temps pour définir les besoins de clients, construire une conception bien correspondante, analyser les fonctions fondamentales, choisir les technologies avant de commencer à entrer dans le domaine de la réalisation et la programmation. Pour éviter des travails inutiles, les discutions avec les clients et les autres managers ou développeurs sont très importantes. 

D'ailleurs, grâce à les projets open sources, il est possible de trouver des projets existants qui réalisent des fonctions intéressantes pour notre projet.