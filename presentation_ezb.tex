\section{Présentation d'eZoomBook}

Le dispositif « eZoomBook » est un outil pédagogique comme la pratique de lecture des jeunes dépendant de l'axe Cultures, imaginaires, médiation de pe. Il est conçu par une équipe d’enseignants-chercheurs en lien avec plusieurs groupes de recherche1 et testé dans le contexte de la formation ingénieur à l’École centrale de Nantes. Au cœur de ce dispositif se trouve la plateforme eZoomBook qui permet de créer et de développer des documents interactifs à échelle multiple en utilisant un système de participation de type « Wiki ». 

Réalisé en 2012, le prototype multidisciplinaire en a été développé et lancé une fois un an après qui était reconnu comme la plateforme eZoomBook. Cependant, la plateforme ne fonctionnait plus à cause des bugs des codes et la perdue de base de données. Les enseignants-chercheurs ont décidé de recommencer le développement qui correspond mieux aux demandes par des améliorations du prototype et des enrichissement des fonctions des codes.
