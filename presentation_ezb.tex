\section{Présentation d'eZoomBook}

Le dispositif « eZoomBook » est un outil pédagogique pour promouvoir la pratique de lecture chez les jeunes. eZoomBook est un programme de recherche qui dépend de l'axe Cultures, imaginaires, médiation de pe. Il est conçu par une équipe d’enseignants-chercheurs en lien avec plusieurs groupes de recherche et testé dans le contexte de la formation ingénieur à l’École centrale de Nantes. Au cœur de ce dispositif se trouve la plateforme eZoomBook qui permet de créer et de développer des documents interactifs à échelle multiple en utilisant un système de participation de type « Wiki ». 

Réalisé en 2012, le prototype multidisciplinaire en a été développé et lancé un an après sous le nom de domaine ezoombook.com. Cependant, le code de la plateforme n'étant pas entretenu, a été remis en question pour, dans sa prochaine étape, ouvrir des perspectives plus larges. Les enseignants-chercheurs ont décidé de recommencer le développement afin de mieux correspondre aux demandes pédagogiques des différentes cadégories d'enseignants.

Par ailleurs, l'équipe a souhaité faire de cet outil un outil open source afin de faciliter les échanges entre chercheurs. Cette désicion a dicté un certain nombre de choix informatiques.