\section{Analyse de l'existant}

La première version de plate-forme d’\ezb\ était lancée une fois par une équipe de développeurs, mais quittée avec des difficultés de la conception et des codes. Quand je suis arrivée, la nouvelle version de la conception et la structure des codes de plate-forme qui s'appelle \textbf{\mini\ } \footnote{Le nom \mini\ qui indique le légèreté de la nouvelle structure ne sert que faire une différence avec l'ancienne version pendant le développement} étaient réalisés par Ophir, un camarade qui travaillait dessus pendant quelques moins. 

Pour me faire connaître des principes ils ont proposés et des technologiques utilisées, pendant le premier moins, j’ai analysé les concepts, choix technologiques et codes de l'\ezb\ et du \mini\ .  

\subsection{Analyse de la conception}
\subsection{Analyse des choix techonologiques}
\subsection{Analyse des fonctionnelles}