\section{Présentation de la \pe\ }

Le projet de recherches pluridisciplinaires EnJeu(x), initié par l'Université d'Angers et le Cerhio, est l'un des lauréats de l'appel à projets « Dynamiques scientifiques », lancé par la Région Pays de la Loire. Durant 5 ans, 130 enseignants-chercheurs de Nantes, Angers et Le Mans vont faire converger leurs connaissances autour d'une thématique : l'enfance et la jeunesse.

Fédérer des acteurs, concentrer des recherches pour leur donner davantage de visibilité au niveau international : c'est la volonté affichée par la Région à travers l'un des trois volets, baptisé « Dynamiques scientifiques », de l'appel à projets Recherche 2014. « L'idée, explique l'historien membre du Cerhio Yves Denéchère, qui a porté le projet EnJeu(x), c'est de regrouper des chercheurs qui, en région, travaillent sur une même thématique, pour faire que dans les 5 ans à venir, cette thématique devienne l'un des points forts de la région, que les Pays de la Loire soient, dans notre cas, reconnus comme un pôle national, voire européen dans le domaine de l'enfance et de la jeunesse ».

Il comprend 5 axes de recherches, découpés en 3 sous-thématiques, ont été définis :

\begin{enumerate}
    \item Développement, éducation, apprentissage
    \begin{itemize}
        \item Aide aux apprentissages et prévention des difficultés
        \item Le rapport des jeunes aux savoirs, à l'école, à l'éducation
        \item Le point de vue des enfants sur leur bien-être, à l'école et en dehors
    \end{itemize}    
    \item Filiations, familles
    \begin{itemize}
        \item Images et mutations familiales  
        \item L'enfant et ses origines (généalogie)
        \item Comparaisons internationales des notions de filiations et d'expressions des parentalités
    \end{itemize}
    \item Cultures, imaginaires, médiation
    \begin{itemize}
        \item La pratique de lecture des jeunes  
        \item Les enjeux du récit
        \item Imaginaires plurilingues
    \end{itemize}
    \item Droits, citoyenneté
    \begin{itemize}
        \item Approches transnationales des droits des enfants 
        \item Institutions, paroles des enfants et des jeunes
        \item Les espaces éducatifs et de loisirs
    \end{itemize}
    \item Enjeux de santé publique
    \begin{itemize}
        \item Maladies de l’enfant, vulnérabilité et handicap  
        \item Santé et environnement numérique
        \item La construction médicale de l'enfance et de l’adolescence
    \end{itemize}
\end{enumerate}