\section{Etudes des vues fonctionnelles}

Pendant le premier mois de l’étude de l’existant, j’ai aussi réalisé les codes de l’édition dans un projet d’ezb de la version d’Ophir. Mais après avoir partagé cette première page d’édition à ec, nous avons constaté que cette proposition n’était pas suffisante pour l’utilisation prévue. De plus, il fallait que la plate-forme reconnaisse différents types d’utilisateurs \footnote{Des enseignants et des élèves au principe} et fasse une distinction entre eux. Cependant, il n’y avait pas un cahier de charge défini des fonctions nécessaires d’ezb. 

Après plusieurs réunions avec \ec\ , \tls\ et les autres enseignants-chercheurs participant au développement d’ezb, on a réussi de définir une liste fonctionnelle qui indiquait bien les fonctions principales et les fonctions supplémentaires :

\begin{enumerate}
    \item La gestion d'utilisateurs 
    \begin{itemize}
        \item Accès permettant d’entrer sur la plateforme avec le compte de l’école
        \item Reconnaissance d’identités entre les enseignants et les élèves
        \item Trois niveaux de droits autorisés aux différents types d’utilisateurs selon leur identité
            \begin{itemize}
                \item Utilisateur Root
                \item Utilisateur Supérieur
                \item Utilisateur Ordinaire
            \end{itemize}
        \item Des messageries distribuées à chaque utilisateur pour des annonces des systèmes
    \end{itemize}
    \item La gestion de livres
    \begin{itemize}
        \item Création de livre à partir des fichiers en multi-format
        \item Définition de droits d’autorisation 
        \item Des autres manipulations nécessaires
        \item Téléchargement de livre en multi-format 
    \end{itemize} 
    \item La gestion de projets
    \begin{itemize}
        \item Création d'un projet à partir d'un projet existant ou d'un livre
        \item Définition de plusieurs finalités notamment une qui permet de couper le livre en plusieurs morceaux et  distribuer les différentes parties aux sous-groupes d’élèves
        \item Contribution collaboration
        \item Manipulations de layers \footnote{Des layers sont des couches sur lesquelles on travaille}
        \item Editions synchronisées dans un layers
        \item Copie d'un projet en choisissant des layers disponibles
        \item Définition d'autorisation pour un groupe mais aussi pour d'autres invités
    \end{itemize}
    \item La gestion de groupes
    \begin{itemize}
        \item Création d'un groupe et choix d'un manager
        \item Modification du groupe
    \end{itemize}   
\end{enumerate}

Pour préciser les fonctions autorisées aux différents type d’utilisateurs, j’ai fait un tableau de définition :
\begin{figure}[H]
\centering
\includegraphics[width=\textwidth]{droits1}
\caption{Droits générals}
\includegraphics[width=\textwidth]{droits2}
\caption{Droits de projets}
\end{figure}
\begin{figure}[H]
\centering
\includegraphics[width=\textwidth]{fonctions}
\caption{Fonctions de projets}
\end{figure}